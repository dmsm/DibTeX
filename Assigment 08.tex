\documentclass[12pt,letterpaper,boxed]{hmcpset}
% set 1-inch margins in the document
\usepackage[margin=1in]{geometry}
\usepackage{marvosym}
\usepackage{MnSymbol,wasysym}
\usepackage{tikz}
\usetikzlibrary{graphs,graphs.standard}
\usepackage{color}
\usepackage{enumerate}
\usepackage{tikz}
\usepackage{mathtools}
\DeclarePairedDelimiter\ceil{\lceil}{\\rceil}
\DeclarePairedDelimiter\\floor{\\lfloor}{\\rfloor}
% include this if you want to import graphics files with /includegraphics
\\name{Assigment 08}
\duedate{2014-11-03}
\\begin{document}
\begin{problem}[Shahriari 8.1.1][20]\\
\\
A coffee company was willing to pay \$1 for each person interviewed about his or her lieks and dislikes on types of cofee. Of the persons interviewed, 270 liked ground coffee, 200 liked instant coffee, 70 liked both, and 50 did not like either choice. What is the total amount of money the company had to pay?

\end{problem}
\begin{problem}[Shahriari 8.1.3][20]\\
\\
An advertising agency finds that of its 170 clients, 115 use television, 100 use radio, 130 use magaines, 75 use television and radio, 95 use radio and magazines, 85 use television and magazines and 70 use all three. How many clients use only magazines? How many client use none of these media?

\end{problem}
\begin{problem}[Shahriari 8.1.4][20]\\
\\
%powers (9883)
How many integers between 1 and 10,000 are neither perfect squares nor perfect cubes.

\end{problem}
\begin{problem}[Shahriari 8.1.6][20]\\
\\
How many permutations of the letters \textbf{SCRIPPS} have no two consecutive letters the same?

\end{problem}
\begin{problem}[Shahriari 8.1.8][20]\\
\\
Lewis Carrol speaks of a battle among 100 combatants in which 80 lost an arm, 85 a leg, 70 an eye and 75 an ear. Some number $p$ of people lost all four. Find the lower and upper bound for $p$.

\end{problem}
\begin{problem}[Shahriari 8.1.10][20]\\
\\
How many five-card hands contain a jack, a queen and king?

\end{problem}
\begin{problem}[Shahriari 8.1.14][20]\\
\\
How many permutations of the 26 letters are there that contain none of the sequences MATH, RUNS, FROM or JOE?

\end{problem}
\begin{problem}[Shahriari 8.1.15][20]\\
\\
Find the number of primes less than 100 without actually finding all the primes

\end{problem}
\begin{problem}[Shahriari 8.2.1][20]\\

Determine the number of 10-combintions of the multiset
$$S = \{ \infty \cdot a, 3 \cdot b, 5 \cdot c, 7 \cdot d \}$$

\end{problem}
\begin{problem}[Shahriari 8.2.3][20]\\
\\
A bakery sells seven kinds of doughnuts. How many ways are there to choose one dozen donuts if no more than three donughts of any kind are used?

\end{problem}
\begin{problem}[Shahriari 8.3.1][20]\\

\item Determine the number of permutations of $\{1,2,\cdots, 8 \}$ in which no even integer is in its natural position.
\item Determine the number of permutations of $\{1,2,\cdots, 8 \}$ in which exactly 4 integers are in their natural position.
\end{enumerate}

\end{problem}
\begin{problem}[Shahriari 8.3.2][20]\\

\item Eight girls are seated around a carousel. In how many ways can they change seats so that each has a different girl in front of her?
\item Eight boys are seated around a carousel but facing inward, so that each boy faces one another. In how many ways can they change seats so that each faces a different boy?
\end{enumerate}

\end{problem}
\begin{problem}[Shahriari 8.3.6][20]\\
\\
For each of the $6 \times 6$ boards with forbidden positions in Figure 8.3 in Shahriari's book on page 147, find the number of ways to place six non-attacking rooks in the non-forbidden positions.

\end{problem}
\begin{problem}[Shahriari 8.3.10][20]\\
\\
Fereydoon, Faranak, Roudabeh, Tahminen and Rostem are students in the class. Each is to be assigned a differnt dialogue form the following list: Crito, Euthyphro, Protagoras, Meno, Parmenides. Fereydoon is not interested in the Crito or the Protagoras; Faranak is not interested in the Meno; Roudabeh is not interested in the Euthyphro or the Parmenides; Tahmineh is not interested in the Euthyphro and Rostam is not interested in the Crito or the Protagoras. In how many ways can we assign the five dialogues to the five students given the above constraints?

\end{problem}
\begin{problem}[Shahriari 8.X.1][20]\\
\\
 Twelve citizens vote for two candidates in a college election. It happens that the vote
is tied 6 to 6. However, for dramatic effect, the ballots are shuffled and read out one at a time
and the cumulative score announced. What is the probability that at some point there is a
gap of 3 votes between the two candidates? Is it closer to 47\% or 48\%?

\end{problem}
\end{document}
